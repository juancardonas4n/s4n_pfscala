\documentclass[handout]{beamer}

\usetheme{default}

\usepackage[latin1]{inputenc}
\usepackage[spanish]{babel}
\usepackage{amssymb,latexsym}
\usepackage{verbatim}
\usepackage{hyperref}
\usepackage{alltt}
\usepackage{url}
\usepackage{listings}
\usepackage{color}
\usepackage{alltt}
\usepackage[ruled,vlined]{algorithm2e}
\usepackage{soul}
\makeatletter
\let\HL\hl
\renewcommand\hl{%
  \let\set@color\beamerorig@set@color
  \let\reset@color\beamerorig@reset@color
  \HL}
\makeatother

\definecolor{mygreen}{rgb}{0,0.6,0}
\definecolor{mygray}{rgb}{0.5,0.5,0.5}
\definecolor{mymauve}{rgb}{0.58,0,0.82}

\lstset{ %
  backgroundcolor=\color{white},   % choose the background color; you must add \usepackage{color} or \usepackage{xcolor}
  basicstyle=\footnotesize,        % the size of the fonts that are used for the code
  breakatwhitespace=false,         % sets if automatic breaks should only happen at whitespace
  breaklines=true,                 % sets automatic line breaking
  captionpos=b,                    % sets the caption-position to bottom
  commentstyle=\color{mygreen},    % comment style
  deletekeywords={...},            % if you want to delete keywords from the given language
  escapeinside={\%*}{*)},          % if you want to add LaTeX within your code
  extendedchars=true,              % lets you use non-ASCII characters; for 8-bits encodings only, does not work with UTF-8
  frame=single,                   % adds a frame around the code
  keepspaces=true,                 % keeps spaces in text, useful for keeping indentation of code (possibly needs columns=flexible)
  keywordstyle=\color{blue},       % keyword style
  language=Octave,                 % the language of the code
  otherkeywords={*,...},           % if you want to add more keywords to the set
  numbers=left,                    % where to put the line-numbers; possible values are (none, left, right)
  numbersep=5pt,                   % how far the line-numbers are from the code
  numberstyle=\tiny\color{mygray}, % the style that is used for the line-numbers
  rulecolor=\color{black},         % if not set, the frame-color may be changed on line-breaks within not-black text (e.g. comments (green here))
  showspaces=false,                % show spaces everywhere adding particular underscores; it overrides 'showstringspaces'
  showstringspaces=false,          % underline spaces within strings only
  showtabs=false,                  % show tabs within strings adding particular underscores
  stepnumber=2,                    % the step between two line-numbers. If it's 1, each line will be numbered
  stringstyle=\color{mymauve},     % string literal style
  tabsize=2,                   % sets default tabsize to 2 spaces
  title=\lstname                   % show the filename of files included with \lstinputlisting; also try caption instead of title
}

\title{Programaci�n Funcional con Scala\\Sesion 3}
\date{18 de Septiembre de 2020}
\author{Juan Cardona}
\institute{S4N}

\begin{document}

\begin{frame}
  \titlepage{}
\end{frame}

\begin{frame}
  \frametitle{Agenda}\tableofcontents
\end{frame}

\section{Unidad 3. Datos inmutables}

\subsection{Introducci�n}
\label{sec:introduccion}

\begin{frame}
  \frametitle{Unidad 3. Datos inmutables}
  \framesubtitle{Introducci�n}
  \begin{block}{Objetivos}

    \begin{itemize}
    \item Definir qu� son los tipos de datos inmutables.
    \item Introducir m�s elementos de la programaci�n funcional pura.
      \begin{itemize}
      \item Coincidencia de patrones
      \item Tipos de datos algebraicos.
      \end{itemize}
    \end{itemize}
  \end{block}

\end{frame}

\subsection{A modo de repaso}
\label{sec:a-modo-repaso}

\begin{frame}
  \frametitle{Unidad 3. Datos inmutables}
  \framesubtitle{A modo de repaso}

  \begin{block}{�Qu� es la programaci�n funcional}
    \begin{quote}
      La programaci�n funcional es una forma de escribir software utilizando �nicamente \hl{funciones puras} y \hl{valores
        inmutables}.
    \end{quote}
  \end{block}
\end{frame}

\subsection{Coincidencia de patrones}
\label{sec:coincidencia-patrones}

\begin{frame}
  \frametitle{Unidad 3. Datos inmutables}
  \framesubtitle{Coincidencia de patrones}

  \begin{itemize}
  \item<1-> Una coincidencia de patrones es una lista de alternativas.
  \item<2-> Cada alternativa es un patr�n o una m�s expresiones
  \end{itemize}  
\end{frame}

% \begin{frame}
%   \frametitle{Unidad 3. Datos inmutables puras}
%   \framesubtitle{Coincidencia de patrones}

%   \begin{block}{Coincidencia de patrones en tuplas}
%     \begin{itemize}
%     \item<1-> Se reconstruye al tupla con el n�mero de posiciones correspondientes.
%     \end{itemize}
%   \end{block}

% \end{frame}

\begin{frame}
  \frametitle{Unidad 3. Datos inmutables}
  \framesubtitle{N�meros fraccionarios}

  \begin{exampleblock}{Ejercicio}
    Seg�n Wikipedia: ``Una \textbf{fracci�n inversa} es una fracci�n
    obtenida a partir de otra dada, en la que se ha invertido el
    numerador y el denominador, es decir la fracci�n $\frac{a}{b}$ se
    convierte en $\frac{b}{a}$.
  \end{exampleblock}

\end{frame}



\subsection{Definici�n de tipos algebraicos}
\label{sec:valores-inmutables}

\begin{frame}
  \frametitle{Unidad 3. Datos inmutables}
  \framesubtitle{Definici�n de tipos algebraicos}

  \begin{exampleblock}{Definici�n de sucesor}
    Los valores n�mericos se pueden definir de una forma
    matem�tica:

    \[
      Numero = \begin{cases}
        0 & \\
        Succ(Numero) &  \\
      \end{cases}
    \]
  \end{exampleblock}
\end{frame}

\begin{frame}[fragile]
  \frametitle{Unidad 3. Datos inmutables}
  \framesubtitle{Definici�n de tipos algebraicos}

  \begin{block}{Definici�n de un valor algebraico}
\begin{lstlisting}[language=Scala]
sealed trait Number
case object Zero extends Number
case class Succ(x:Number) extends Number
\end{lstlisting}    
  \end{block}
\end{frame}

\begin{frame}[fragile]
  \frametitle{Unidad 3. Datos inmutables}
  \framesubtitle{Definici�n de tipos algebraicos}

  \begin{exampleblock}{Definici�n}
\begin{lstlisting}[language=Scala]
def isZero(x:Number):Boolean = x match {
case Zero => true
case _    => false
}
\end{lstlisting}    
  \end{exampleblock}
\end{frame}

\begin{frame}
  \frametitle{Unidad 3. Datos inmutables}
  \framesubtitle{Definici�n de tipos algebraicos}

  \begin{enumerate}
  \item Defina una funci�n (\texttt{sum}) se sume dos n�meros.
  \item Defina una funci�n (\texttt{mult}) que multiplique dos
    n�meros.
  \end{enumerate}

\end{frame}

\end{document}

%%% Local Variables:
%%% mode: latex
%%% TeX-master: t
%%% End:
